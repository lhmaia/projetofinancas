\begin{resumo}

    No mercado de ações há duas categorias de métodos de análise que são utilizados como ferramentas pelos investidores para alcançar maiores ganhos, são a análise técnica e a análise fundamentalista. Enquanto na análise fundamentalista destaca-se o estudo de uma empresa e de seu valor de mercado, na análise técnica o que vale é o estudo do movimento dos preços no mercado ao longo do tempo. Na análise técnica os investidores utilizam indicadores, baseados na  observação do comportamento do mercado, para prever tendências.

Este trabalho tem por objetivo estabelecer relações entre informações publicadas na mídia e em redes sociais com o desempenho de ações na bolsa de valores. A partir dos dados extraídos e quantificados vamos procurar estabelecer relações com séries históricas de preços de ações. As relações estabelecidas poderão culminar na criação de um indicador a ser utilizado em operações futuras na bolsa. Além das notícias serão realizados experimentos com indicadores conhecidos, em separado e em conjunto com os dados de notícias coletados.



    \textbf{Palavras-chave:} Finanças, computação, mineração de dados, indicadores, análise técnica, svm, naive bayes.
\end{resumo}
