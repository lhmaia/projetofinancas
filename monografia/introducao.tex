\chapter{INTRODUÇÃO}

\section{Motivação}

    Um dos fatores que influenciam a flutuação dos preços no mercado de açoes são as informações disponíveis ao público \cite{ChanAndChui2001}. Notícias divulgadas na imprensa e postagens em redes sociais podem provocar movimentos de elevação ou queda no preço de um ativo. 

O mercado financeiro é um elemento crucial no desenvolvimento das sociedades modernas. Através do mercado financeiros agentes superavitários, que desejam investir seus recursos podem alcançar agentes deficitários dispostos a empreender. As boas alternativas de investimento, com possibilidade de grandes retornos tem levado um público cada vez maior a buscar o mercado de ações e consequentemente um maior número de pesquisadores tem voltado sua atenção para esse assunto.

No entanto a maior parte das pesquisas se concentra em abordagens da análise técnica. O que se justifica pela anseio dos investidores em ter acesso a métodos quantitativos e a disponibilidade de grandes bases históricas de preços \cite{Nassirtoussi2014}. Informações textuais, disponíveis em notícias de jornais, revistas e redes sociais são uma fonte de dados importante para avaliação do desempenho do mercado, entretando devido a natureza não estruturada dessas fontes de dados normalmente são utilizadas apenas em abordagens fundamentalistas. O tratamento dessas informações através de métodos computacionais e seu uso na análise do comportamento de mercado é uma área de pesquisa desafiadora. 


\section{Objetivos}

Neste trabalho tentaremos quantificar a relevância das informações publicadas e tentar estabelecer uma relação entre uma notícia veiculada e a variação no preço de uma ação.

Para alcançar os objetivos propostos utilizaremos técnicas de mineração de dados. Trabalharemos com algoritmos como SVM e Naive Bayes, conhecidos como algoritmos supervisionados. Eles são chamados dessa forma por se caracterizarem por trabalhar em duas etapas: a etapa de treinamento, no qual aprendem um padrão treinando sobre uma base de dados conhecida e classificadas e outra etapa de teste no qual tentamos fazer previsões sobre outra base.

Nossas bases de dados são constituídas por séries históricas de preços de ações, obtidas a partir do serviço Bloomberg Professional, contratado pelo CEFET/MG e dados históricos de notícias conseguidos através do Observatório da Web, da UFMG.
