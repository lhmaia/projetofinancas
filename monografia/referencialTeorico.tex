
\chapter{REFERENCIAL TEÓRICO}

\section{Trabalhos Relacionados}
Nesta seção analisamos alguns trabalhos relevantes em mineração de dados voltados para a aplicação no mercado financeiro.

Em \cite{Mittermayer2004} é descrito o NewsCATS, um sistema que automaticamente analisa e classifica em categorias noticias da imprensa e partir dessa classificação provê recomendações para negociações na bolsa de valores.

Tendo objetivo selecionar uma estratégia de investimento que possa alcançar uma rentabilidade acima da média de mercado, o NewsCATS tenta prever tendencias nos preços das ações no momento imediatamente apos a publicação de notícias na imprensa. Ele trabalha com tres componentes, o primeiro utiliza tecnicas de pré-processamento para recuperar informações relevantes, o segundo classifica as nformações em categorias e o terceiro deriva, a partir das informações fornecidas pelos outros dois componentes, estratégias para investimento.

O sistema foi implementado em JAVA, são utilizadas técnicas de pré-processamento como remoção de stop-words e steeming e a represenção das características dos textos é feita utilizando modelos como TF (term-frequency), IDF (Inverse Document Frequency) e TF x IDF. O algoritmo utilizado na classificação das notícias é o SVM.

Colocar resultados do NewsCATS

Um exemplo de trabalho utilizando redes sociais está presente em \cite{SulDennisYuan2014}. Nesse artigo os pesquisadores confrontam dados coletados no Twitter com retorno diário das ações de empresas que fazem parte do índice S\&P 500. Esse trabalho estabelece uma medida das emoções expressas nos tweets em relação a uma determinada empresa com o retorno das ações dessa empresa. O artigo trabalha com o conceito de valência emocional, podendo a valência ser positiva ou negativa.

O artigo investiga o período pelo qual as informações publicadas no Twitter permanecem influenciando um ativo e a relevância do número de seguidores de um usuário na influência que ele exerce.

A partir das análises os autores chegaram a conclusão que os tweets influenciam as ações dentro do mesmo dia e em um intervalo de 10 dias. Além disso usuários com um número de seguidores acima da média são capazes de exercer uma influência mais imediata.

O trabalho de \cite{SchumakerChen2010} aborda a utilização de uma síntese de tecnicas de linguística, finanças e estatística para criar o Arizona Financial Text System (AZFinText), um sistema para previsao de preços discretos de ações.

O AZFinText utiliza uma combinação entre os preços das ações e o texto de notícias financeiras para fazer previsões a respeito do preço discreto de uma ação em um período curto de tempo. O sistema confia na hipótese de que há uma pequena janela de tempo que o mercado leva para encontrar o equilíbrio após uma nova informação ser noticiada, dessa forma uma abordagem automatizada é capaz de explorar essa janela e conseguir lucros acima da média do mercado.

Os autores utilizaram o algoritmo SVR, uma versão do SVM que utiliza regressão. As entradas para o algoritmo são as notícias e o preço atual dos ativos e o alvo é o preço a ser previsto. Para avaliar o retorno o AZFinText foi executado por um ano e obteve um retorno de 8,5% contra 5,62% do índice S&P 500.

\section{Algoritmos Supervisionados}

Os algoritmos supervisionados caracterizam-se por utilizar uma base de dados de treino, na qual para cada elemento da base o valor alvo é conhecido, ou seja, dada uma questão a ser respondida pelo algoritmo, a base de dados de treino consiste em um conjunto de pares de entradas e saídas.

O algoritmo supervisionado analisa essa base de treino e a partir dela busca inferir uma função a partir da qual possam ser estabelecidas as saídas para novas entradas.

Podemos enumerar algumas etapas básicas que devem ser realizadas na aplicação de algoritmos supervisionados:

\begin{enumerate}
	\item Determinar um conjunto de exemplos de treinamento para ser utilizado. Este conjunto deve ser representativo para modelar o problema a ser resolvido. Deve ser conseguido um conjunto de entradas e determinadas suas respectivas saídas, em alguns casos será necessário fazer um trabalho manual para conseguir um conjunto classificado.
	\item Definir uma representação das características para as entradas de dados da função de aprendizagem. O mais comum é representar uma entrada como um vetor de características.
	\item Escolher o algoritmo de  aprendizagem mais adequado.
	\item Dependendo da escolha do algoritmo, algumas parametros são necessários para o processamento. Ajustar esses parametros é fundamental para um bom resultado. Uma técnica utilizada para chegar a valores de parametros satisfatórios é a validação cruzada.
	\item Avaliar a acurácia da função de aprendizagem em um conjunto de testes.
\end{enumerate}


Nos experimentos realizados neste trabalho foram utilizados dois algoritmos supervisionados, o SVM e o Naive Bayes.

\subsection{SVM - Support Vector Machine}

A técnica de aprendizado de máquina SVM - Support Vector Machine - constitui um campo de pesquisa importante entre os algoritmos supervisionados. O SVM apresenta resultados superiores a outros algoritmos supervisionados, como as redes neurais, quando aplicado a campos como mineração de textos e reconhecimentos de imagens.

Dado um conjunto de vetores de treinamento, onde os valores alvo dentro de cada vetor pertencem a duas classes, sendo essas classes linearmente separaveis, o SVM é capaz de encontrar um hiperplano que melhor separa os conjuntos de vetores nas suas devidas classes.

Mais formalmente podemos dizer que dado Xi para i em {1, 2, …, N} os vetores do conjunto de treinamento. Esses vetores pertencem às classes W1 e W2, que assumimos como classes linearmente separaveis. O objetivo do SVM é encontrar um hiperplano g(x) = WTx + w0 = 0 que separa corretamente os vetores.

Há situações em que os vetores não podem ser separados linearmente, nesses casos é aplicada uma transformação. Essa transformação consiste em uma função não-linear, conhecida como função kernel, que mapeia as entradas do conjunto de treinamento em uma dimensão na qual é possível encontrar um hiperplano que separe corretamente os vetores para suas respectivas classes.

As funções kernel mais utilizadas são:


	%linear: K(xi, xj) = xT i xj.
	%• polynomial: K(xi, xj) = (γxiTxj + r)d, γ > 0.
	%• radial basis function (RBF): K(xi, xj) = exp(−γkxi − xjk2), γ > 0.
	%• sigmoid: K(xi, xj) = tanh(γxiTxj + r).
	%Here, γ, r, and d are kernel parameters.



O tipo de kernel e os parâmetros a serem utilizados variam de acordo com a natureza do problema no qual estão sendo aplicados.

\subsection{Naive Bayes}

Os algoritmos Naive Bayes constituem uma família de algoritmos baseadas na aplicação do Teorema de Bayes. O nome naive se deve ao fato do algoritmo assumir a hipótese de independência entre as características do problema a ser tratado.

A aplicação do naive bayes na categorização de textos é bastante comum. O seu uso se justifica pela simplicidade do algoritmo e os bons resultados apresentados, em muitos casos alcançando um resultado melhor que abordagens mais complexas.

As implementações do naive bayes variam e sua utilização depende da distribuição probabilística do conjunto de dados no qual será aplicado. Algumas implementações são naive Bayes multinomial, gaussiano e bernoulli.
