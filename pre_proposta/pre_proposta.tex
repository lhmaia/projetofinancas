%%%%%%%%%%%%%%%%%%%%%%%%%%%%%%%%%%%%%%%%%
% Journal Article
% LaTeX Template
% Version 1.3 (9/9/13)
%
% This template has been downloaded from:
% http://www.LaTeXTemplates.com
%
% Original author:
% Frits Wenneker (http://www.howtotex.com)
%
% License:
% CC BY-NC-SA 3.0 (http://creativecommons.org/licenses/by-nc-sa/3.0/)
%
%%%%%%%%%%%%%%%%%%%%%%%%%%%%%%%%%%%%%%%%%

%----------------------------------------------------------------------------------------
%	PACKAGES AND OTHER DOCUMENT CONFIGURATIONS
%----------------------------------------------------------------------------------------

\documentclass[twoside]{article}

\usepackage{lipsum} % Package to generate dummy text throughout this template
\usepackage[brazil]{babel}
\usepackage[utf8]{inputenc}

\usepackage[sc]{mathpazo} % Use the Palatino font
%\usepackage[T1]{fontenc} % Use 8-bit encoding that has 256 glyphs
\linespread{1.05} % Line spacing - Palatino needs more space between lines
\usepackage{microtype} % Slightly tweak font spacing for aesthetics

\usepackage[hmarginratio=1:1,top=20mm, bottom=20mm, left=15mm,columnsep=15pt]{geometry} % Document margins
\usepackage{multicol} % Used for the two-column layout of the document
\usepackage[hang, small,labelfont=bf,up,textfont=it,up]{caption} % Custom captions under/above floats in tables or figures
\usepackage{booktabs} % Horizontal rules in tables
\usepackage{float} % Required for tables and figures in the multi-column environment - they need to be placed in specific locations with the [H] (e.g. \begin{table}[H])
\usepackage{hyperref} % For hyperlinks in the PDF

\usepackage{lettrine} % The lettrine is the first enlarged letter at the beginning of the text
\usepackage{paralist} % Used for the compactitem environment which makes bullet points with less space between them

\usepackage{abstract} % Allows abstract customization
\renewcommand{\abstractnamefont}{\normalfont\bfseries} % Set the "Abstract" text to bold
\renewcommand{\abstracttextfont}{\normalfont\small\itshape} % Set the abstract itself to small italic text

\usepackage{titlesec} % Allows customization of titles
\renewcommand\thesection{\Roman{section}} % Roman numerals for the sections
\renewcommand\thesubsection{\Roman{subsection}} % Roman numerals for subsections
\titleformat{\section}[block]{\large\scshape\centering}{\thesection.}{1em}{} % Change the look of the section titles
\titleformat{\subsection}[block]{\large}{\thesubsection.}{1em}{} % Change the look of the section titles

\usepackage{fancyhdr} % Headers and footers
\pagestyle{fancy} % All pages have headers and footers
\fancyhead{} % Blank out the default header
\fancyfoot{} % Blank out the default footer
\fancyhead[C]{Cefet MG / UFMG $\bullet$ Abril 2016} % Custom header text
\fancyfoot[RO,LE]{\thepage} % Custom footer text

%----------------------------------------------------------------------------------------
%	TITLE SECTION
%----------------------------------------------------------------------------------------

\title{\vspace{-15mm}\fontsize{10pt}{10pt}\selectfont\textbf{Relação entre Publicações na Imprensa e em Redes Sociais com o Desempenho de Ações na Bolsa de Valores - Pré-projeto}} % Article title
\author{
\normalsize Luiz Henrique Maia Cruz \\ % Your name
\normalsize Cefet MG / UFMG \\ % Your institution
\vspace{-5mm}
}
\date{}

%----------------------------------------------------------------------------------------

\begin{document}

\maketitle % Insert title

\thispagestyle{fancy} % All pages have headers and footers

%----------------------------------------------------------------------------------------
%	ARTICLE CONTENTS
%----------------------------------------------------------------------------------------

\begin{multicols}{2} % Two-column layout throughout the main article text

\section{Introdução}

\lettrine[nindent=0em,lines=3]{E} ste trabalho tem por objetivo estabelecer relações entre informações publicadas na mídia e em redes sociais com o desempenho de ações na bolsa de valores.

As informações disponíveis ao público influenciam a tomada de decisão dos investidores e portanto tem reflexo no preço das ações \cite{ChanChui:2001}. Neste projeto tentaremos quantificar a relevância das  informações divulgadas, considerando a fonte da notícia, o volume de veículos de informação que a divulgaram e a repercussão em redes sociais.

A partir dos dados extraídos e quantificados vamos procurar estabelecer relações com séries históricas de preços de ações. As relações estabelecidas poderão culminar na criação de um indicador a ser utilizado em operações futuras na bolsa.

Para abordar esse problema utilizaremos técnicas de mineração de dados como algoritmos de classificação e agrupamentos. Técnicas de análise de sentimento serão utilizadas para classificar a opinião expressa nas diversas fontes de notícias. No contexto do mercado de ações, classificar se uma notícia é positiva ou negativa, significa inferir se o conteúdo daquela notícia é capaz de interferir positiva ou negativamente no preço daquela ação \cite{ChanChui:2001}.

Além de artigos, notícias e postagens extraídas de portais de notícias e redes sociais, serão utilizados no projeto um conjunto de dados extraído do serviço \emph{Bloomberg Professional}. Esse serviço disponibiliza em tempo real notícias e análises do mercado financeiro.

%------------------------------------------------

\section{Trabalhos Relacionados}

Em \cite{Mittermayer:2004} é descrito o \emph{NewsCATS}, um sistema para prever tendências nos preços das ações no momento imediatamente após a publicação de notícias na imprensa. \emph{NewsCATS} trabalha com três componentes, o primeiro utiliza técnicas de pré-processamento para recuperar informações relevantes, o segundo classifica as informações em categorias e o terceiro deriva, a partir das informações fornecidas pelos outros dois componentes, estratégias para investimento.

O trabalho de \cite{SchumakerChen:2010} aborda a utilização de uma síntese de técnicas de linguística, finanças e estatística para criar o \emph{Arizona Financial Text System}, um sistema para previsão de preços discretos de ações.

Um exemplo de trabalho utilizando redes sociais está presente em \cite{SulDennisYuan:2014}. Nesse trabalho os autores utilizam informações do \emph{Twitter} para avaliar a influência da disseminação de informações, dessa rede social, sobre o preço das ações na bolsa.


%------------------------------------------------

\section{Desenvolvimento}

O desenvolvimento deste trabalho inclui as seguintes etapas:
\begin{enumerate}
\item Levantamento de dados históricos de ações, notícias da imprensa e postagens em redes sociais.
\item Definição das técnicas de pré-processamento e representação dos dados a serem utilizadas.
\item Escolha do algorítimo a ser utilizado para levantar as possíveis relações entre os dados da bolsa e as notícias.
\item Estudo das relações levantadas para tentar derivar um indicador.
\item Teste do indicador sobre outra massa de dados
\item Avaliação dos resultados.
\end{enumerate}


%----------------------------------------------------------------------------------------
%	REFERENCE LIST
%----------------------------------------------------------------------------------------

\begin{thebibliography}{99} % Bibliography - this is intentionally simple in this template

\bibitem[Chan and Chui, 2001]{ChanChui:2001}
Chan, Y., Chui, A., \& Kwok, C. (2001).
\newblock The impact of salient political and economic news on the trading activity.
\newblock {\em Pacific-Basin Finance Journal}, Volume 9, Issue 3, June 2001, Pages 195–217.

\bibitem[Mittermayer, 2004]{Mittermayer:2004}
Mittermayer, M.-A., C. (2004).
\newblock Forecasting Intraday Stock Price Trends with Text Mining Techniques.
\newblock {\em Proceedings of the 10th Annual Hawaii International Conference on System Sciences. Big Island,
Hawaii: IEEE Computer Society}.

\bibitem[Schumaker and Chen, 2010]{SchumakerChen:2010}
Schumaker, R. P., \& Chen, H. , C. (2010).
\newblock A Discrete Stock Price Prediction Engine Based on Financial News.
\newblock {\em Computer}, Vol. 43, No. 1 , 51-56.

\bibitem[Sul and Dennis and Yuan, 2014]{SulDennisYuan:2014}
Sul, H. K., Dennis, A. R., Yuan, L. (2014).
\newblock Trading on Twitter: The Financial Information Content of Emotion in Social Media
\newblock {\em Hawaii International Conference on System Science}, 2014 47th.
 
\end{thebibliography}

%----------------------------------------------------------------------------------------

\end{multicols}

\end{document}
